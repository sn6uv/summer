\documentclass{article}

\usepackage{amsmath}
\usepackage{amssymb}
\usepackage{amsopn}
\usepackage{amsthm}
\usepackage{hyperref}
\usepackage{geometry}

\newcommand{\R}{\mathbb{R}}

\newtheorem{lemma}{Lemma}\newcommand{\lemmaautorefname}{Lemma}
\newtheorem{cond}{Condition}\newcommand{\condautorefname}{Condition}

\title{Inclusion Bounds}

\begin{document}
\maketitle
\begin{cond}
\label{cond1}
The domain $\Omega \subset \R^n$ is bounded, with piecewise smooth boundary in the sense that
\[
\Omega = \bigcap_{i=1}^N\{x : f_i(x) > 0\}
\]
where the $f_i$ are smooth functions defined on a neighbourhood of $\bar\Omega$ such that
\begin{itemize}
\item $\nabla f_i \ne 0$ on the set $\{f_i = 0\}$,
\item $\{f_i = f_j = 0\}$ is an embedded submanifold of $\R^n$, $1 \le i < j \le N$, and
\item $\Omega$ is locally Lipschitz.
\end{itemize}
\end{cond}

\begin{lemma}
\label{lem22}
Let $\Omega \subset \R^n$ satisfy \autoref{cond1}.
Then there exits a smooth vector field $a$ defined on a neighbourhood of $\bar\Omega$ such that
\[
a \cdot n \ge 1
\]
almost everywhere on $\partial \Omega$.
\end{lemma}

\begin{lemma}
\label{lem21}
Let $\Omega \subset \R^n$ satisfy \autoref{cond1}.
Let $E > 1$ and let
\[
\phi = \sum_{|E_j - E| \le E^{1/2}} c_j \phi_j
\]
where $c_j \in \R$ and $\phi_j \subset L^2(\Omega)$ are (orthonormal) eigenfunctions associated with the Laplacian with eigenvalues $E_j$ and satisfy the Dirichlet boundary condition.
Assume that $\lVert c_j \rVert_{\ell^2} = 1$.
Then
\[
\lVert \partial_n \phi\rVert_{L^2(\Omega)}^2 \le C_\Omega E,
\]
where the constant $C$ depdends only on $\Omega$.
\end{lemma}
\begin{proof}
We begin by deriving an identity for first order differential operators.
Greens second identity gives
\begin{align*}
\int_{\partial \Omega} (D \phi) \partial_n \phi \, ds
& = \int_{\partial \Omega} \phi \, \partial_n (D \phi) ds + \int_{\Omega}  \Big((D \phi)\Delta \phi - (\Delta (D \phi)) \phi \Big) dx \\
& = \int_{\Omega} \Big((D \phi)(\Delta \phi) - \phi (\Delta D \phi)\Big) dx \\
& = \int_{\Omega} \Big((D \phi) (\Delta \phi) - \phi [\Delta, D] \phi + \phi (D \Delta \phi) \Big) dx \\
& = \int_{\Omega} \Big((D \phi) (\Delta + E) \phi - \phi [\Delta, D] \phi + \phi (D (\Delta + E) \phi) \Big) dx.
\end{align*}
The second equality is using the fact that $\phi$ satisfies the Dirichley boundary condition.
We then apply the definition of the commutator and finish by adding and subtracting $\phi E (D \phi)$.
% TODO well definedness
We pick $D = a \cdot \nabla$ where $a$ comes from \autoref{lem22}.
We thus notice
\[
\lVert \partial_n \phi \rVert_{L^2(\partial \Omega)}^2
\le \int_{\partial \Omega} (a \cdot n) (\partial_n \phi)^2 ds
= \int_{\partial \Omega} (D \phi) \partial_n \phi ds.
\]
By using the above identity we thus deduce,
\begin{equation}
\label{eq-3terms}
\lVert \partial_n \phi \rVert_{L^2(\partial \Omega)}^2 \le \int_{\Omega} (D \phi) (\Delta + E) \phi dx - \int_{\Omega} \phi [\Delta, D] \phi dx + \int_{\Omega} \phi (D (\Delta + E) \phi) dx.
\end{equation}
We want to bound each of these three terms.
Defining $C_a = \sup_{x \in \Omega} |a(x)|$ we first have
\begin{align*}
\lVert D \phi \rVert_{L^2(\Omega)}^2 \le C_a^2 \int_{\Omega} \lVert \nabla \phi\rVert^2 dx & = - C_a^2 \int_{\Omega} \phi \Delta \phi dx \\
& = -C_a^2 \int_{\Omega} \phi \sum_{|E_j - E| \le E^{1/2}} c_j \Delta \phi_j dx
\\
& = C_a^2 \sum_{|E_j - E| \le E^{1/2}} c_j^2 E_j \\
& \le C_a^2 \underbrace{\sum_{|E_j - E| < E^{1/2}} c_j^2}_1 (E + E^{1/2})
\end{align*}
The first equality is an integration by parts using the fact that $\phi$ satisfies the Dirichlet boundary condition.
The rest follows from the fact that $\phi$ is normalised and has an expansion in terms of eigenfunctions $\phi_j$ with associated eigenvalues $-E_j$.
Furthermore,
\begin{align*}
\lVert D(\Delta + E)\phi\rVert_{L^2(\Omega)}^2 & \le C_a^2 \int_{\Omega} \lvert \nabla (\Delta + E) \phi\rvert^2 dx \\
& = -C_a^2 \int_{\Omega} \sum_{i,j} ((\Delta + E) \phi_i) (\Delta (\Delta + E) \phi_j) dx \\
& = C_a^2 \int_{\Omega} c_i c_j \sum_{i,j} ((-E_i + E) \phi_i) E_j (-E_j + E) \phi_i) dx \\
& = C_a^2 \sum_j c_j^2 E_j (E-E_j)^2
\le C_a^2 E^{3/2}
\le C_a^2 E (E + E^{1/2})
\end{align*}
The first equality is an integration by parts. We then express $\phi$ is the basis of (orthogonal) eigenfuncions.
The final estimate we use is similar
\[
\lVert (\Delta + E) \phi \rVert_{L^2(\Omega)}^2
= \int_{\Omega} \Big((\Delta + E)\sum_j c_j \phi_j\Big)^2 dx
= \sum_j c_j^2 (E-E_j)^2 \le E.
\]
The Cauchy-Schwartz inequality thus allows us to bound the last two terms in \eqref{eq-3terms}
\[
\int_{\Omega} (D \phi) (\Delta + E)\phi dx \le \lVert D\phi \rVert_{L^2(\Omega)} \lVert (\Delta + E) \phi \rVert_{L^2(\Omega)}
\le C_a \sqrt{E(E + E^{1/2})},
\]
\[
\int_{\Omega} \phi D(\Delta + E) \phi dx \le \lVert D(\Delta + E) \phi \rVert_{L^2(\Omega)} \le C_a \sqrt{E (E + E^{1/2})},
\]
so that their sum is bounded by $2 C_a \sqrt{E (E+E^{1/2})}$.
It remains to estimate the first term on the right hand side of \eqref{eq-3terms}.
We get
\begin{align*}
- \int_{\Omega} [\Delta, D] \phi dx
& = -\int_{\Omega} \partial_{ii} (a_j \partial_j \cdot) - a_j \partial_{iij} \phi dx \\
& = % TODO
\end{align*}
\end{proof}

\end{document}
