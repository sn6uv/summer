\documentclass[10pt]{article}

\usepackage{amsmath}
\usepackage{amssymb}
\usepackage{url}

\newcommand{\R}{\mathbb{R}}
\newcommand{\C}{\mathbb{C}}
\newcommand{\N}{\mathbb{N}}

\title{Pseudo-differential operators}
\author{Angus Griffith}

\begin{document}

\maketitle

\section{Introduction}
Pseudo-differential operators are a generialisation of differential operators.
\paragraph{}
Let $\Omega \subset \R^n$ and consider a linear differential operator of order $m$ with constant coefficients
\[
p(D) = \sum_{|\alpha| \le m} a_\alpha D^\alpha,
\]
where $\alpha \in \N^n$ are multiindices and $a_\alpha \in \C$.
The symbol of such an operator is simple to compute,
\[
P(\xi) = \sum_{|\alpha| \le m} a_\alpha \xi^\alpha.
\]
\paragraph{}
Let $f : \Omega \to \C$.
We can solve the PDE
\[
P(D) u = f
\]
by applying the Fourier transform to both sides
\[
p(\xi) \hat{u}(\xi) = \hat{f}(\xi).
\]
Provided the symbol does not vanish in $\R^n$ we rearrange to find
\[
\hat{u}(\xi) = \frac{1}{P(\xi)} \hat{f}(\xi).
\]
Since $P(\xi)$ is continuous and non-vanishing, $\frac{1}{P} \in L^\infty(\R^n)$ and so we can invert the Fourier transform to find a solution
\[
u(x) = \frac{1}{(2 \pi)^n} \int_{\R^n} e^{i x \cdot \xi} \frac{1}{P(\xi)} \hat{f}(\xi) d\xi.
\]
Expanding the definition of $\hat{f}$,
\[
u(x) = \frac{1}{(2 \pi)^n} \int_{\R^{2n}} e^{i (x-y) \cdot \xi} \frac{1}{P(\xi)} f(y) \, dy \, d\xi.
\]
Note that we are assuming $u$ and $f$ have well defined Fourier transforms, e.g. $u,f \in L^2$, $u,f \in \mathcal{S}'$ etc.
\paragraph{}
Natural questions to ask:
\begin{itemize}
\item What is the meaning of assuming that our solution has a well defined Fourier transform?
\item What classes of non-constant coefficients $a_\alpha(x)$ could we hope to handle?
\end{itemize}
\section{Pseudo-differential operators}


%
\begin{thebibliography}{9}
\bibitem{gerard07}
\emph{Pseudo-differential Operators and the Nash--Moser Theorem},
Serge Alinhac and Patrick G\'erard,
Graduate Studies in Mathematics vol. 82,
\textbf{2007}.

\bibitem{wiki}
\emph{Pseudo-differential operator},
Wikipedia, the free encyclopedia,
\url{http://en.wikipedia.org/wiki/Pseudo-differential_operator},
Accessed: Jan 2015.
\end{thebibliography}
\end{document}
