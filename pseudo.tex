\documentclass[10pt]{article}

\title{Pseudo-differential operators}
\author{Angus Griffith}

\usepackage{amsmath}
\usepackage{amssymb}
\usepackage{url}
\usepackage{enumerate}

\newcommand{\R}{\mathbb{R}}
\newcommand{\C}{\mathbb{C}}
\newcommand{\N}{\mathbb{N}}

\newtheorem{defn}{Definition}
\newtheorem{ex}{Example}

\begin{document}

\maketitle

\section{Differential Operators}
Pseudo-differential operators are a generialisation of differential operators.
\paragraph{}
Let $\Omega \subset \R^n$ and consider a linear differential operator of order $m$ with constant coefficients
\[
p(D) = \sum_{|\alpha| \le m} a_\alpha D^\alpha,
\]
where $\alpha \in \N^n$ are multiindices and $a_\alpha \in \C$.
The symbol of such an operator is simple to compute,
\[
P(\xi) = \sum_{|\alpha| \le m} a_\alpha \xi^\alpha.
\]
\paragraph{}
Let $f : \Omega \to \C$.
We can solve the PDE
\[
P(D) u = f
\]
by applying the Fourier transform to both sides
\[
p(\xi) \hat{u}(\xi) = \hat{f}(\xi).
\]
Provided the symbol does not vanish in $\R^n$ we rearrange to find
\[
\hat{u}(\xi) = \frac{1}{P(\xi)} \hat{f}(\xi).
\]
Since $P(\xi)$ is continuous and non-vanishing, $\frac{1}{P} \in L^\infty(\R^n)$ and so we can invert the Fourier transform to find a solution
\[
u(x) = \frac{1}{(2 \pi)^n} \int_{\R^n} e^{i x \cdot \xi} \frac{1}{P(\xi)} \hat{f}(\xi) d\xi.
\]
Expanding the definition of $\hat{f}$,
\[
u(x) = \frac{1}{(2 \pi)^n} \int_{\R^{2n}} e^{i (x-y) \cdot \xi} \frac{1}{P(\xi)} f(y) \, dy \, d\xi.
\]
Note that we are assuming $u$ and $f$ have well defined Fourier transforms, e.g. $u,f \in L^2$, $u,f \in \mathcal{S}'$ etc.
\paragraph{}
Natural questions to ask:
\begin{itemize}
\item What is the meaning of assuming that our solution has a well defined Fourier transform?
\item What classes of non-constant coefficients $a_\alpha(x)$ could we hope to handle?
\end{itemize}
\section{Symbols}
If we applied the above analysis to the differential operator
\[
P = \sum_{|\alpha| \le m} a_\alpha(x) D^{\alpha}
\]
where $a \in \mathcal{S}$ we would obtain the formula
\begin{align*}
Pu(x) & = (2 \pi)^{-n} \int_{\R^n} e^{ix \cdot \xi} p(x,\xi) \hat{u}(\xi) d\xi, \\
p(x,\xi) & = \sum_{|\alpha| \le m} a_\alpha(x) \xi^\alpha.
\end{align*}
Notice that this definition still makes sense more $p$ more general than just a polynomial.
We can generalise $p$ to any `reasonable' function $a = a(x,\xi)$.
The corresponding operator $a(x,D)$ will be said to be the pseudo-differential operator with symbol $a(x,\xi)$ and will be denoted $A$.
By reasonable we mean that $a(x,\xi)$ there exists $m$ such that for all $\alpha$,
\begin{enumerate}[(i)]
\item
$\displaystyle |\partial_\xi^\alpha a(x,\xi)| \le C_\alpha (1 + |\xi|)^{m-|\alpha|}$,
\item
$\displaystyle |\partial_x^\alpha a(x,\xi)| \le C_\alpha (1 + |\xi|)^m$.
\end{enumerate}
The first condition captures that $a(x,\xi)$ should have polynomial type behaviour as $|\xi| \to \infty$, improving with respect to differentiation with respect to $\xi$.
The second condition expresses that the amplitude $a(x,\xi)$ and phase $e^{i x \cdot \xi}$ can be distinguished.
\paragraph{}
We formaise this further with the following definition
\begin{defn}
Let $m \in \R$. Let $S^m = S^m(\R^n \times \R^n)$ denote the set of $a \in C^\infty(\R^n \times \R^n)$ such that
\[
\forall \alpha, \beta, \ |\partial_x^\alpha \partial_\xi^\beta a(x,\xi)| \le C_{\alpha,\beta} (1 + |\xi|)^{m-|\beta|}.
\]
An element $a \in S^m$ is called a symbol of order $m$.
\end{defn}
Note the similarity to the definition of Schwartz functions.
\begin{defn}
If $a(x,\xi) = \sum_{|\alpha| \le m} a_\alpha(x) \xi^\alpha$ with $a_\alpha \in \C^\infty(\R^n)$ is bounded, along with all its derivatives
and $a \in S^m$ then $a$ is said to be a differential symbol.
\end{defn}
As one might expect, $S^m$ forms a Fr\'echet space which can be equipped with semi-norms
\[
\lVert \alpha \rVert_{\alpha, \beta}^m = \sup_{(x,\xi) \in \R^n \times \R^n} \left\{(1 + |\xi|)^{-m - |\beta|} \big| \partial_x^\alpha \partial_\xi^\beta a(x,\xi)\big| \right\}.
\]
\section{Pseudo-differential operators}
We now define the pseudo-differential operators,
\begin{defn}
For $a \in S^m$,
\[
Op(a) : u(x) \mapsto (2\pi)^{-n} \int_{\R^n} e^{i x \cdot \xi} a(x,\xi) \hat{u}(\xi) d\xi
\]
is the pseudo-differential operator with symbol $a$.
A pseudo-differential operator is said to be of order $m$ if its symbol belongs to $S^m$.
\end{defn}


%
\begin{thebibliography}{9}
\bibitem{gerard07}
\emph{Pseudo-differential Operators and the Nash--Moser Theorem},
Serge Alinhac and Patrick G\'erard,
Graduate Studies in Mathematics vol. 82,
\textbf{2007}.

\bibitem{wiki}
\emph{Pseudo-differential operator},
Wikipedia, the free encyclopedia,
\url{http://en.wikipedia.org/wiki/Pseudo-differential_operator},
Accessed: Jan 2015.
\end{thebibliography}
\end{document}
