\documentclass{article}

\usepackage{amsmath}
\usepackage{amssymb}
\usepackage{amsopn}

\newcommand{\R}{\mathbb{R}}

\title{Layer Potentials}

\begin{document}
\maketitle

Let $\Omega \subset \R^n$ be a bounded domain with smooth boundary.
Fix $f : \partial \Omega \to \R$ and consider Laplace's equation
\begin{equation}
\label{eq-laplace}
\begin{cases}
\Delta u = 0 & \text{in } \Omega, \\
u = f & \text{on } \partial \Omega.
\end{cases}
\end{equation}
where $\Delta = \sum_{j=1}^n \partial_{jj}$ is the Laplacian.
We define the fundamental solution of Laplace's equation,
\begin{equation}
\label{eq-fund_soln_laplace}
\Gamma(x-y)
% = \Gamma(|x-y|)
= \begin{cases}
\frac{1}{2 \pi} \log|x-y|, & n = 2,  \vspace{2mm} \\
\displaystyle \frac{-1}{n(n-2)V_{n} |x-y|^{n-2}}, & n \ge 3,
\end{cases}
\end{equation}
where $V_n$ is the volume of the unit ball in $\R^n$.
We compute      % TODO Show steps
\begin{align*}
\partial_i \Gamma(x-y) & = \frac{1}{n V_n} \frac{x_i - y_i}{|x-y|^n}, \\
\partial_{ij} \Gamma(x-y) & = \frac{1}{n V_n} \frac{|x-y|^2 \delta_{ij} - n (x_i - y_i)(x_j - y_j)}{|x-y|^{n+2}}.
\end{align*}
Note the decay estimates,
\begin{align*}
|\partial_i \Gamma(x-y)| & \lesssim_n \frac{1}{|x-y|^{n-1}}, \\
|\partial_{ij} \Gamma(x-y)| & \lesssim_n \frac{1}{|x-y|^{n}},
\end{align*}
for all $x \ne y$.
We wish to obtain the Green's identities.
For all $u,v \in C^2(\bar\Omega)$, an integration by parts yields Greens first identity
\[
\int_{\Omega} v \Delta u \, dx = \int_{\partial \Omega} v \frac{\partial u}{\partial \nu} ds - \int_{\Omega} Du \cdot Dv dx.
\]
Interchanging $u$ and $v$ and subtracting gives Green's second identity,
\[
\int_{\Omega} (v \Delta u - u \Delta v) dx = \int_{\partial \Omega} \left(v \frac{\partial u}{\partial \nu} - u \frac{\partial v}{\partial \nu}\right)ds.
\]
We cannot just simply substitute $\Gamma$ in place of $v$ in this formula due to the singularity at $x = y$.
We first consider $\Omega \setminus B(y, \rho)$ where $\rho > 0$ is chosen to be small.
We have Green's second identity on this subdomain,
\[
\int_{\Omega \setminus B(y,\rho)} \Gamma \Delta u dx = \int_{\partial \Omega} \left(\Gamma \frac{\partial u}{\partial \nu} - u \frac{\partial \Gamma}{\partial \nu}\right) ds + \int_{\partial B(y,\rho)} \left(\Gamma \frac{\partial u}{\partial \nu} - u \frac{\partial \Gamma}{\partial \nu}\right) ds.
\]
Now, since $\Gamma$ is radially symmetric,
\[
\int_{\partial B(y,\rho)} \Gamma \frac{\partial u}{\partial \nu} ds = \Gamma(\rho) \int_{\partial B(y,\rho)} \frac{\partial u}{\partial \nu} ds
\lesssim \rho^{n-1} \sup_{B(y,\rho)}|Du| \to 0 \quad \text{as}\ \rho \to 0.
\].
Likewise,using the above expression for $\partial_i \Gamma$,
\[
\int_{\partial B(y,\rho)} u \frac{\partial \Gamma}{\partial \nu} ds = -\Gamma'(\rho) \int_{\partial B(y,\rho)} u ds = \frac{-1}{n V_n \rho^{n-1}} \int_{\partial B(y,\rho)} u ds \to u(y)
\quad \text{as}\ \rho \to 0.
\]
Letting $\rho$ tend to zero, we derive Green's representation formula,
\begin{equation}
\label{eq-green_rep}
u(y) = \int_{\partial \Omega} \left( u \frac{\partial \Gamma}{\partial \nu} (x-y) - \Gamma(x-y) \frac{\partial u}{\partial \nu} \right)ds + \int_{\Omega} \Gamma(x-y) \Delta u \, dx
\qquad (y \in \Omega).
\end{equation}
In particular, a harmonic function $v \in C^1(\bar\Omega) \cap C^2(\Omega)$ can be represented by its boundary values,
\[
v(y) = \int_{\partial \Omega} \left( v \frac{\partial \Gamma}{\partial \nu} (x-y) - \Gamma(x-y) \frac{\partial v}{\partial \nu} \right)ds \qquad (y \in \Omega).
\]
We thus define the single-layer potential and double layer potentials respectively, $S,D : L^2(\partial \Omega) \to L^2(\partial \Omega)$ by
\[
S(v) = \int_{\partial \Omega}\Gamma(x-y) \frac{\partial v}{\partial \nu} ds, \qquad D(v) = \int_{\partial \Omega} v \frac{\partial \Gamma}{\partial \nu} (x-y) ds.
\]

\begin{thebibliography}{99}
\bibitem{evans}
Partial Differential Equations,
Lawrence C. Evans,
\textbf{2001}

\bibitem{gt}
Elliptic Partial Differential Equations of Second Order,
Gilbarg, David and Trudinger, Neil,
Second Edition,
\textbf{1983}

\end{thebibliography}
\end{document}
